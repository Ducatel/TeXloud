\documentclass[a4paper,12pt]{article}

% paquets pour avoir les lettres accentués et la typographie française
\usepackage{a4wide}
\usepackage[utf8]{inputenc}
\usepackage[frenchb]{babel}

%\usepackage{glossaries}



\usepackage[colorlinks=true,linkcolor=black]{hyperref}
\usepackage[T1]{fontenc}%la police utilisee dans le document

%trois package pour taper du texte mathémathiques
\usepackage{amsmath}
\usepackage{amssymb}
\usepackage{amsfonts}
\usepackage{mathrsfs}

%pour insérer des styles de liste supplémentaire
%\usepackage{enumerate}

%pour l'insertion d'image
\usepackage{graphicx}
\usepackage{float}% pour l'utilisation de "centering" et de l'option [H] qui permet de placer plus efficacement les images

%package pour les lien url :
\usepackage{url}


%pour l'insertion de code source


\usepackage{fancyvrb}

\begin{document}

%%%%%%%%%%%%%%%%%%%%%%%%%%%%%%%%%%%%%%%%%%%%%%%%%%%%
% PAGE DE GARDE
%%%%%%%%%%%%%%%%%%%%%%%%%%%%%%%%%%%%%%%%%%%%%%%%%%%%

\begin{titlepage}
\begin{flushleft}
\large{Universit\'e du Havre \\
Master Matis \\
Sp\'ecialisation SIRES\\
}
\end{flushleft}

\setlength{\parskip}{96pt}

\begin{center}
\huge\textbf{TeXloud\\Des documents \LaTeX ~dans le Cloud}

\setlength{\parskip}{18pt}
\large\textsc{Référent: Y. Pigné}

\setlength{\parskip}{70pt}

\Large\textbf{Rapport}

\setlength{\parskip}{50pt}

\large Adrien Bruyère\\David Ducatel\\Meva Rakotondratsima\\Sidina Biha\\Zakaria Bouchakor
\end{center}
\setlength{\parskip}{50pt}
\begin{flushleft}
\rule{.4mm}{26mm}\rule{105mm}{.4mm}
\today
\end{flushleft}
\end{titlepage}

%%%%%%%%%%%%%%%%%%%%%%%%%%%%%%%%%%%%%%%%%%%%%%%%%%%%
% FIN DE LA PAGE DE GARDE
%%%%%%%%%%%%%%%%%%%%%%%%%%%%%%%%%%%%%%%%%%%%%%%%%%%%
 
\clearpage

\tableofcontents

\newpage

\section{Introduction}
\subsection{Rappel du sujet}
\paragraph*{}

\subsection{Rappel de l'équipe}
\paragraph*{}

\section{Choix techniques}
\paragraph*{}

\section{Utilisation standard}
\subsection{Création de compte}
\paragraph*{}

\subsection{Création de projet}
\paragraph*{}
\begin{itemize}
 \item Authentification
 \item Creation d'un projet
 \item Creation d'un fichier/dossier
 \item sync
\end{itemize}


\subsection{Compilation}
\paragraph*{}
\begin{itemize}
 \item Lancement compilation
 \item log
 \item visu du pdf
 \item telechargement du pdf
\end{itemize}


\subsection{Fonction annexe}
\paragraph*{}
\begin{itemize}
 \item Suppression de dossier/fichier
 \item Rename dossier/fichier
 \item Info supp sur user
\end{itemize}

\section{Perspective d'évolution}
\subsection{Support des images}
\paragraph*{}

\subsection{Gestion des groupes}
\paragraph*{}

\subsection{Coloration syntaxique sur Android}
\paragraph*{}

\subsection{Intégration de PDFJS}
\paragraph*{}



\end{document}
