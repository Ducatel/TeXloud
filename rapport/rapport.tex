\documentclass[a4paper,12pt]{article}

% paquets pour avoir les lettres accentués et la typographie française
\usepackage{a4wide}
\usepackage[utf8]{inputenc}
\usepackage[frenchb]{babel}

%\usepackage{glossaries}



\usepackage[colorlinks=true,linkcolor=black]{hyperref}
\usepackage[T1]{fontenc}%la police utilisee dans le document

%trois package pour taper du texte mathémathiques
\usepackage{amsmath}
\usepackage{amssymb}
\usepackage{amsfonts}
\usepackage{mathrsfs}

%pour insérer des styles de liste supplémentaire
%\usepackage{enumerate}

%pour l'insertion d'image
\usepackage{graphicx}
\usepackage{float}% pour l'utilisation de "centering" et de l'option [H] qui permet de placer plus efficacement les images

%package pour les lien url :
\usepackage{url}


%pour l'insertion de code source


\usepackage{fancyvrb}

\begin{document}

%%%%%%%%%%%%%%%%%%%%%%%%%%%%%%%%%%%%%%%%%%%%%%%%%%%%
% PAGE DE GARDE
%%%%%%%%%%%%%%%%%%%%%%%%%%%%%%%%%%%%%%%%%%%%%%%%%%%%

\begin{titlepage}
\begin{flushleft}
\large{Universit\'e du Havre \\
Master Matis \\
Sp\'ecialisation SIRES\\
}
\end{flushleft}

\setlength{\parskip}{96pt}

\begin{center}
\huge\textbf{TeXloud\\Des documents \LaTeX ~dans le Cloud}

\setlength{\parskip}{18pt}
\large\textsc{Référent: Y. Pigné}

\setlength{\parskip}{70pt}

\Large\textbf{Rapport}

\setlength{\parskip}{50pt}

\large Adrien Bruyère\\David Ducatel\\Meva Rakotondratsima\\Sidina Biha\\Zakaria Bouchakor
\end{center}
\setlength{\parskip}{50pt}
\begin{flushleft}
\rule{.4mm}{26mm}\rule{105mm}{.4mm}
\today
\end{flushleft}
\end{titlepage}

%%%%%%%%%%%%%%%%%%%%%%%%%%%%%%%%%%%%%%%%%%%%%%%%%%%%
% FIN DE LA PAGE DE GARDE
%%%%%%%%%%%%%%%%%%%%%%%%%%%%%%%%%%%%%%%%%%%%%%%%%%%%
 
\clearpage

\tableofcontents

\newpage

\section{Introduction}
\subsection{Rappel du sujet}
\paragraph*{}
Ce projet propose la création et la gestion collaborative de documents
Latex. Le but est de proposer à des plateformes dépourvues de distribution
Latex (tablettes, smartphones, desktops), de se connecter au Web et
d'accéder à ces service de gestion et de compilation de documents.
\paragraph*{} 
Les utilisateurs seront authentifiés au service et bénéficieront d'un espace
de stockage privé. L'application facilitera le partage de documents et le
travail collaboratif entre utilisateurs du service.
\paragraph*{} 
Coté client, deux types d'applications seront développés :

\begin{itemize}
 \item Un service Web permettra l'accès au service à partir de n'importe
quelle machine (desktop, tablette non-Android) pourvue d'un navigateur
Web et d'une connexion internet

 \item Une application Android, permettra une certaine autonomie avec le
stockage temporaire d'une copie de travail des documents, permettant
un mode d'édition non-connecté.
\end{itemize}

\subsection{L'équipe}
\paragraph*{}
L'équipe est composé de 5 personnes:
\begin{itemize}
 \item Adrien Bruyère est responsable du développement de l'application Android
 \item David Ducatel (chef de projet) est responsable du développement de la frontale et du service de compilation.
 \item Meva Rakotondratsima est responsable du développement du service de stockage de données.
 \item Sidina Biha, Zakaria Bouchakor sont responsables du développement de l'application WEB
\end{itemize}

\newpage
\section{Choix techniques}
\subsection{Architecture général}
\paragraph*{}

\subsection{Service web}
\paragraph*{}

\subsection{Application Android}
\paragraph*{}

\subsection{Service de routage et d'ordonnancement}
\paragraph*{}

\subsection{Service de stockage de données}
\paragraph*{}

\subsection{Service de compilation}
\paragraph*{}

\newpage
\section{Utilisation des applications}
\subsection{Application Web}
\subsubsection{Création de compte}
\paragraph*{}

\subsubsection{Création de projet}
\paragraph*{}

\subsubsection{Compilation}
\paragraph*{}
\begin{itemize}
 \item Lancement compilation
 \item log
 \item visu du pdf
 \item telechargement du pdf
\end{itemize}


\subsubsection{Fonctions annexes}
\paragraph*{}
\begin{itemize}
 \item Suppression de dossier/fichier
 \item Rename dossier/fichier
 \item Info supp sur user
\end{itemize}

\subsection{Application Android}
\subsubsection{Création de compte}
\paragraph*{}

\subsubsection{Création de projet}
\paragraph*{}

\subsubsection{Compilation}
\paragraph*{}
\begin{itemize}
 \item Lancement compilation
 \item log
 \item visu du pdf
 \item telechargement du pdf
\end{itemize}


\subsubsection{Fonctions annexes}
\paragraph*{}
\begin{itemize}
 \item Suppression de dossier/fichier
 \item Rename dossier/fichier
 \item Info supp sur user
\end{itemize}

\newpage
\section{Perspective d'évolution}
\subsection{Support des images}
\paragraph*{}

\subsection{Gestion des groupes}
\paragraph*{}

\subsection{Coloration syntaxique sur Android}
\paragraph*{}

\subsection{Intégration de PDFJS}
\paragraph*{}



\end{document}
