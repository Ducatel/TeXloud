\documentclass[a4paper,12pt]{article}

% paquets pour avoir les lettres accentu�s et la typographie fran�aise
\usepackage{a4wide}
\usepackage[latin1]{inputenc}
\usepackage[frenchb]{babel}
\usepackage[T1]{fontenc}%la police utilisee dans le document

%trois package pour taper du texte math�mathiques
\usepackage{amsmath}
\usepackage{amssymb}
\usepackage{amsfonts}
\usepackage{mathrsfs}

%pour ins�rer des styles de liste suppl�mentaire
%\usepackage{enumerate}

%pour l'insertion d'image
\usepackage{graphicx}
\usepackage{float}% pour l'utilisation de "centering" et de l'option [H] qui permet de placer plus efficacement les images

%package pour les lien url :
\usepackage{url}


%pour l'insertion de code source


\usepackage{fancyvrb}

\begin{document}

%%%%%%%%%%%%%%%%%%%%%%%%%%%%%%%%%%%%%%%%%%%%%%%%%%%%
% PAGE DE GARDE
%%%%%%%%%%%%%%%%%%%%%%%%%%%%%%%%%%%%%%%%%%%%%%%%%%%%

\begin{titlepage}
\begin{flushleft}
\large{Universit\'e du Havre \\
Master seconde ann\'ee \\
Informatique sp\'ecialisation SIRES\\
}
\end{flushleft}

\setlength{\parskip}{96pt}

\begin{center}
\huge\textbf{TeXloud}

\setlength{\parskip}{18pt}
\large\textsc{R�f�rent: Y. Pign�}

\setlength{\parskip}{70pt}

\Large\textbf{Cahier des charges}

\setlength{\parskip}{50pt}

\large Adrien Bruy�re\\David Ducatel\\Meva Rakotondratsima\\Sidina Biha\\Zakaria Bouchakor
\end{center}
\setlength{\parskip}{50pt}
\begin{flushleft}
\rule{.4mm}{26mm}\rule{105mm}{.4mm}
\today
\end{flushleft}
\end{titlepage}

%%%%%%%%%%%%%%%%%%%%%%%%%%%%%%%%%%%%%%%%%%%%%%%%%%%%
% FIN DE LA PAGE DE GARDE
%%%%%%%%%%%%%%%%%%%%%%%%%%%%%%%%%%%%%%%%%%%%%%%%%%%%
 
\clearpage

\tableofcontents

\newpage

\section{Introduction}
	
D�crire bri�vement l'environnement dans lequel s'inscrit le projet (strat�gie, enjeux, domaine, etc.)

\section{Description de la demande}

\subsection{Les objectifs}
 	
D�finir les r�sultats que le projet doit atteindre.
 	
M�thode : Un �nonc� d'objectif doit comporter un verbe d'action � l'infinitif et un objet.
 	
Exemple : Diffuser un corpus de connaissance assimilable par toute personne de niveau Bac + 4.
\subsection{Produit du projet}
 	
Proposer une description g�n�rale de ce produit.
\subsection{Les fonctions du produit}
 	
Lister et justifier les principales fonctionnalit�s du produit.
\subsection{Crit�res d'acceptabilit� et de r�ception}
 	
Formuler des indicateurs pr�cis qui permettent de mesurer si les objectifs de performance du produit sont atteints.
 	
Exemple : Le produit doit r�pondre � la norme XX0.


\section{Contraintes}

\subsection{Contrainte de d�lais}
 	
Sp�cifier la date de livraison du produit et les �ventuelles �ch�ances interm�diaires.
\subsection{Contraintes technique}
 	
Sp�cifier les �ventuelles autres contraintes � prendre en compte dans le cadre du projet (normes techniques, clauses juridiques, etc.)

\section{D�roulement du projet}

\subsection{Planification}
 	
Repr�senter l'articulation des grandes phases du projet et des principaux jalons.

\subsection{Ressources}
 	
Lister les ressources humaines et mat�rielles que le client peut mettre � la disposition du prestataire.


\end{document}