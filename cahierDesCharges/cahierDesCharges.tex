\documentclass[a4paper,12pt]{article}

% paquets pour avoir les lettres accentués et la typographie française
\usepackage{a4wide}
\usepackage[utf8]{inputenc}
\usepackage[frenchb]{babel}
\usepackage[T1]{fontenc}%la police utilisee dans le document

%trois package pour taper du texte mathémathiques
\usepackage{amsmath}
\usepackage{amssymb}
\usepackage{amsfonts}
\usepackage{mathrsfs}

%pour insérer des styles de liste supplémentaire
%\usepackage{enumerate}

%pour l'insertion d'image
\usepackage{graphicx}
\usepackage{float}% pour l'utilisation de "centering" et de l'option [H] qui permet de placer plus efficacement les images

%package pour les lien url :
\usepackage{url}


%pour l'insertion de code source


\usepackage{fancyvrb}

\begin{document}

%%%%%%%%%%%%%%%%%%%%%%%%%%%%%%%%%%%%%%%%%%%%%%%%%%%%
% PAGE DE GARDE
%%%%%%%%%%%%%%%%%%%%%%%%%%%%%%%%%%%%%%%%%%%%%%%%%%%%

\begin{titlepage}
\begin{flushleft}
\large{Universit\'e du Havre \\
Master Matis \\
Sp\'ecialisation SIRES\\
}
\end{flushleft}

\setlength{\parskip}{96pt}

\begin{center}
\huge\textbf{TeXloud\\Des documents \LaTeX ~dans le Cloud}

\setlength{\parskip}{18pt}
\large\textsc{Référent: Y. Pigné}

\setlength{\parskip}{70pt}

\Large\textbf{Cahier des charges}

\setlength{\parskip}{50pt}

\large Adrien Bruyère\\David Ducatel\\Meva Rakotondratsima\\Sidina Biha\\Zakaria Bouchakor
\end{center}
\setlength{\parskip}{50pt}
\begin{flushleft}
\rule{.4mm}{26mm}\rule{105mm}{.4mm}
\today
\end{flushleft}
\end{titlepage}

%%%%%%%%%%%%%%%%%%%%%%%%%%%%%%%%%%%%%%%%%%%%%%%%%%%%
% FIN DE LA PAGE DE GARDE
%%%%%%%%%%%%%%%%%%%%%%%%%%%%%%%%%%%%%%%%%%%%%%%%%%%%
 
\clearpage

\tableofcontents

\newpage

\section{Introduction}
\paragraph*{}
LaTeX est un langage de composition de documents créé en 1983, dédié principalement à la rédaction
 de documents scientifiques, dont les éléments sémantiques sont définis par des mots-clés 
(définition de paragraphes, titres...). Il permet d'écrire simplement des formules scientifiques 
(équations mathématiques), et l'organisation des documents est gérée automatiquement (pagination, 
etc.).

\paragraph*{}
Le cloud computing repose sur le principe de délocalisation des traitements informatiques
 traditionnellement localisés sur des serveurs locaux ou sur le poste client de l'utilisateur. Cela 
permet une meilleure répartition des charges systèmes et des tâches.

\section{Description de la demande}

\subsection{Produit du projet}
 	
\paragraph*{} 
Ce projet propose la création et la gestion collaborative de documents
Latex. Le but est de proposer à des plateformes dépourvues de distribution
Latex (tablettes, smartphones, desktops), de se connecter au Web et
d’accéder à ces service de gestion et de compilation de documents.
\paragraph*{} 
Les utilisateurs seront authentifiés au service et bénéficieront d’un espace
de stockage privé. L’application facilitera le partage de documents et le
travail collaboratif entre utilisateurs du service.
\paragraph*{} 
Coté client, deux types d’applications seront développés :\\

\begin{itemize}
 \item Un service Web permettra l’accès au service à partir de n’importe
quelle machine (desktop, tablette non-Androïd) pourvue d’un navigateur
Web et d’une connexion internet

 \item Une application Android, permettra une certaine autonomie avec le
stockage temporaire d’une copie de travail des documents, permettant
un mode d’édition non-connecté.
\end{itemize}
\newpage
\subsection{Les fonctions du produit}
 	
Les fonctionnalités principales du produit sont les suivantes :\\

\begin{itemize}
 \item FP0 - \'Edition des documents Latex
 \item FP1 - Accès à l'ensemble des projets
 \item FP2 - Interface Web et Android
 \item FP3 - Compilation Latex
 \item FP4 - Création de compte
 \item FP5 - Authentification
 \item FP6 - Téléchargement des documents
 \item FP7 - Synchronisation des documents
 \item FP8 - Charte graphique
\end{itemize}
\bigskip
Fonctionnalités complémentaires :\\
\begin{itemize}
 \item FC0 - Versioning (gestion de conflits, etc.)
 \item FC1 - Gestion des groupes d'utilisateurs
 \item FC2 - Gestion des erreurs
 \item FC3 - Unification des interfaces
 \item FC4 - Sauvegarde locale (travail offline)
\end{itemize}

\paragraph{FP0 - \'Edition des documents Latex\\}
La fonctionnalité principale de l'interface (Android ou Web) est l'édition de documents Latex. 
L'éditeur sera la zone principale de l'application, afin de pouvoir afficher le plus de texte 
possible.

\paragraph{FP1 - Accès à l'ensemble des projets\\}
Lors de l'authentification, l'utilisateur récupère l'arborescence de ses projets (dossiers, 
fichiers). Le fichier selectionné sera ensuite téléchargé, et l'utilisateur pourra travailler.

\paragraph{FP2 - Interface Web et Android\\}
L'utilisateur a deux possibilités pour se connecter à TeXloud, \emph{via} :\\
\begin{itemize}
 \item Interface web : on peut se connecter de n'importe quel poste (ordinateur personnal, cybercafé...)
 \item Système Android : connexion à partir d'une tablette Android. L'utilisateur pourra alors 
travailler par un réseau Wifi ou 3G, ou bien en local (offline).
\end{itemize}

\paragraph{FP3 - Compilation Latex\\}
Lorsque l'utilisateur souhaite avoir un rendu PDF de son document Latex, il doit pouvoir demander 
la compilation au serveur.

\paragraph{FP4 - Création de compte\\}
Pour qu'un utilisateur puisse utiliser le service TeXloud, une inscription est nécessaire. La 
création de compte est faisable par l'application android ou l'application web. L'utilisateur doit
fournir un nom, un mot de passe et une adresse mail.

\paragraph{FP5 - Authentification\\}
A chaque démarrage de l'application, l'utilisateur envoie son login et mot de passe. L'authentification
est nécessaire pour pouvoir utiliser l'application TeXloud.

\paragraph{FP6 - Téléchargement des documents\\}
Le téléchargement des documents est une fonctionnalité capitale de l'application. Les documents 
compilés (PDF) doivent pouvoir être envoyés à l'utilisateur, ainsi que les fichiers Latex, pour que 
l'utilisateur travaille toujours sur la version la plus récente.

\paragraph{FP7 - Synchronisation des documents\\}
Un fichier Latex en cours de modification doit être régulièrement synchronisé avec le serveur.

\paragraph{FP8 - Charte graphique\\}
Définir une charte graphique.


\paragraph{FC0 - Versioning (gestion de conflits, etc.)\\}
L'application devra intégrer un gestionnaire de version, afin de permettre un meilleur travail de
groupe.

\paragraph{FC1 - Gestion des groupes d'utilisateurs\\}
Plusieurs personnes peuvent travailler sur un même projet. Chaque projet a donc un ou plusieurs 
utilisateurs qui ont le droit de modifier les fichiers. Le créateur du projet doit former le groupe.

\paragraph{FC2 - Gestion des erreurs\\}
Le client doit pouvoir voir les différentes erreurs, afin de les corriger. Par exemple : erreur 
de compilation, conflit de version sur un fichier.

\paragraph{FC3 - Unification des interfaces\\}
L'interface des deux applications doit être similaire. L'utilisateur doit pouvoir retrouver ses
repères rapidement, en passant d'une application à l'autre. Par exemple : position et ordre des 
éléments, onglets, thème graphique.

\paragraph{FC4 - Sauvegarde locale (travail offline)\\}
Une tablette Android peut à tout moment perdre sa connexion Internet (voyage, zone non couverte). 
Il est donc important de pouvoir travailler hors-ligne à partir d'un fichier temporaire.

\subsection{Critères d'acceptabilité et de réception}
 	
Le projet pourra être considéré comme complet, si les critères ci-dessous sont remplis :
\begin{itemize}
 \item L'interface web est entièrement développée en HTML5 et validée par le W3C
 \item Les services de compilation et de stockage de données sont distribués sur plusieurs machines
 \item Les interfaces Android et Web sont unifiées
 \item L'application est utilisable
\end{itemize}


\section{Contraintes}

\subsection{Contrainte de délais}

Audits intermédiaires :
\begin{itemize}
 \item 09 Décembre 2011
 \item 06 Janvier 2012
 \item 13 Janvier 2012
\end{itemize}

Rendu de projet : Courant Fevrier 2012

\subsection{Contraintes technique}
 	
Spécifier les éventuelles autres contraintes à prendre en compte dans le cadre du projet (normes techniques, clauses juridiques, etc.)

\section{Déroulement du projet}

\subsection{Planification}
\subsubsection{Diagramme de planification}
\includegraphics[width=1\textwidth,angle=90]{mpm.pdf}
\newpage
\subsubsection{Definition de l'environnement de développement\\}

\paragraph{Environnement materiel\\} 
Nous avons besoin de 7 serveurs (virtualisation possible) et une tablette Android. Les serveurs sont :
\begin{itemize}
 \item Serveur Web, bases de données et mail
 \item Serveur frontal de gestion de données
 \item Deux serveurs de stockage des données
 \item Serveur frontal de compilation
 \item Deux serveurs de compilation
\end{itemize}

\bigskip

\paragraph{Environnement logiciel\\}
Voici les environnements logiciels utilisés :
\begin{itemize}
 \item Serveur http : Apache
 \item SGBD : PostgreSQL
 \item GIT et SVN pour le versioning et stockage de données
 \item Interpreteur PHP 5  
 \item Compilateur Latex
 \item Serveur SMTP Postfix
 \item Hyperviseur (si virtualisation)
 \item Système d'exploitation Android
\end{itemize}
 

\paragraph{Langages de programmation\\}
Liste des langages de programmation utilisés :
\begin{itemize}
 \item HTML 5, CSS 3
 \item PHP 5
 \item Javascript (jQuery)
 \item Python
 \item Java + XML (Android)
 \item SQL
\end{itemize}
\newpage
\subsubsection{Installation de serveurs\\}
L'étape consiste à installer tout l'environnement logiciel cité précédemment. 

\subsection{Ressources}
 	
Ressources humaines : 5 étudiants. Ressources matérielles : Serveur, Tablette ??


\end{document}